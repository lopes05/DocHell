\chapter{Processo de engenharia de requisitos}

\section{SAFe 4.0}

O processo que o grupo irá utilizar foi baseado no Scaled Agile Framework (SAFe) em sua versão 4.0, o SAFe é um framework que fornece padrões para implementar a metodologia Ágil em nível de empresas, provendo uma visão que guia em seus três níveis: Portfólio, Programa e Time.

\begin{figure}[!htpb]
\centering
\includegraphics[scale=0.6]{figuras/processo/SAFe_Big_Picture_4}
\caption{Big Picture que representa a atual versão do SAFe}
\end{figure}

\subsection{Nível de Portfólio}
O nível de portfólio consiste no nível mais alto de abstração do SAFe, esse nível fornece construções básicas em torno do fluxo de valor do negócio para guiar empresas que adotam a metodologia ágil.

\subsection{Nível de Programa}
O SAFe detalha o nível de programa como o nível onde são detalhadas as Features e os incrementos de programas (PI), contendo uma das atividades primordiais do framework, o PI Planning.

\subsection{Nível de Time}

O nível de time é o nível mais baixo no framework, nele são aplicados os rituais do Scrum, onde as histórias de usuário são criadas a partir das features para serem desenvolvidas em Sprints.

\section{Papéis utilizados}

\subsection{Nível de Portfólio}
Os papéis que vão ser utilizados no nível de portfólio são os Epic Owners e o Gerente de Programa de Portfólio. No SAFe 4.0 há mais um papel nesse nível que é o de Arquiteto Empresarial, que serve para dar uma visão de um nível mais alto em relação a aspectos arquiteturais, além de guiar os times em torno de uma visão comum,  como o projeto vai trabalhar com apenas uma equipe, e não é algo muito complexo, esse papel não será utilizado no processo.

\begin{itemize}
\item Epic Owners: O Epic Owner tem a responsabilidade de guiar os Épicos individualmente pelo Processo do Kanban de Portfólio e influeniar na decisão do Gerente de Programa de Portfólio, quando o épico é aceito, ele trabalha com os times de desenvolvimento e o gerente de produto para iniciar as atividades necessárias para atender o épico.
\item Gerente de Programa de Portfólio: Segundo o SAFe, esse papel representa as pessoas que tem o nível mais alto de estratégia para realizar a tomada de decisões no framework SAFe. Tem o papel de participar da definição  e comunicação dos Temas Estratégicos, além de definir e priorizar o Backlog de Épicos.
\end{itemize}

\subsection{Nível de Programa}
Os papéis a serem utilizados no nível de programa são o de gerente de produto, e arquiteto de sistema, os outros papéis nesse nível não serão utilizados por se tratar de um projeto pequeno que envolve apenas uma equipe de quatro pessoas.

\begin{itemize}
\item Gerente de produto: É uma das autoridades principais contidas no SAFe, guia o nível de Programa, cria o Visão, e trabalha com o cliente para entender o que ele precisa, além de definir, priorizar e validar requisitos. 
\item Arquiteto do sistema: Serve para gerar uma visão comum entre os times de desenvolvimento e a solução a ser desenvolvida. Ele também participa da definição de sistemas e subsistemas, validam questões envolvendo tecnologia e avaliam alternativas. Participam das definições de nível mais alto na definição dos requisitos não-funcionais.
\end{itemize}

\subsection{Nível de Time}
Os papéis a serem utilizados no nível de time são os mesmos utilizados no SCRUM, sendo eles o Scrum Master, o Product Owner e o time de desenvolvimento.

\begin{itemize}
\item Scrum Master: O papel de Scrum Master é assumido por um membro do time, que vai ter a responsibilidade de ajudar o time a ser organizar e atingir seus objetivos, implementando e dando suporte aos princípios e práticas do SCRUM e do SAFe, isso tudo para manter um fluxo correto na realização das atividades.
\item Product Owner: É o membro do time que tem o papel de definir as histórias de usuários e seus critérios de aceitação, além de priorizar o Backlog das mesmas, enquanto mantém a integridade conceitual e técnica das features e componentes de responsabilidade do time.
\item Time de Desenvolvimento: Consiste em uma equipe pequena onde essas pessoas são as responsáveis pelo desenvolvimento de código e testes para a aplicação, além disso participam de todas as atividades no nível de time.
\end{itemize}

\newpage
\section{Processo elaborado pela equipe}
\begin{figure}[!htpb]
\centering	
\includegraphics[scale=0.45]{figuras/processo/modelo}
\caption{Modelo do processo a ser utilizado}
\end{figure}

\newpage

\section{Atividades}

\subsection{Atividade: Definir Temas Estratégicos}
\begin{itemize}
\item Participantes: Gerente de Programa de Portfólio, Product Owner
Descrição: De acordo com o SAFe, os temas estratégicos são objetivos de negócio específicos que servem para conectar o Portfólio à estratégia de negócio da empresa. No contexto desse trabalho, essa atividade vai praticamente seguir a definição fornecida pelo SAFe, para alinhar o trabalho proposto com a estratégia de negócio que o grupo vai adotar. 

\item Subatividades(s): 
    \begin{itemize}
    \item Levantar Temas Estratégicos: Consiste em reuniões com o cliente que estará assumindo o papel de Product Owner, para estabelecer quais objetivos terão de ser alcançados.
    \item Analisar Temas Estratégicos: Fazer uma análise dos temas estratégicos levantados para garantir que os stakeholders compreenderam os mesmos.
    \item Documentar Temas Estratégicos:  Gerar um documento contendo os Temas Estratégicos coletados, essa subatividade contará com o auxílio da Ferramenta de Gerenciamento de Requisitos.
    \item Validar Temas Estratégicos: Tem o intuito de garantir que os temas estratégicos documentados realmente atendem ao problema especificado pelo cliente, para então prosseguir para a próxima atividade.
    \end{itemize}
	Artefatos de entrada:
    Não são utilizados artefatos de entrada nesta atividade.
 
\item Artefato(s) Gerado(s) e/ou Alterado(s): Temas Estratégicos documentados na Ferramenta de Gerenciamento de Requisitos.
\item Atividade(s) Predecessora(s): Como essa atividade é a primeira do processo, não há atividades que a precedem.
\item Atividade(s) Sucessora(s): Definir épicos

\end{itemize}

\subsection{Atividade: Definir épicos}
\begin{itemize}
\item Participantes: Epic owners, Product Owner

\item Descrição: De acordo com o SAFe, os épicos conseguem transmitir as maiores iniciativas de projeto localizadas no nível Portfólio, os épicos de negócio agregam valor de uma forma direta no projeto, sendo compreendidos de investimento intenso e tem um impacto significativo no projeto. 

\item Subatividades(s): 
	\begin{itemize}
  	\item Levantar Épicos: esse levantamento também será executado em conjunto do Product Owner, para definir quais problemas devem ser resolvidos e elaborar uma visão mais completa das necessidades do cliente, além de elaborar os critérios de aceitação para esses épicos.
    \item Analisar Épicos: Fazer uma análise dos épicos levantados para garantir que os stakeholders envolvidos compreenderam os épicos levantados e suas necessidades, além de alertar possíveis problemas.
    \item Documentar Épicos:  Documentar os épicos levantados na ferramenta de Gerenciamento de Requisitos.
    \item Validar Épicos: Garantir que os épicos levantados são diretamente voltados a resolução do problema apontado pelo cliente.
    \end{itemize}
    
\item Artefatos de entrada: Temas estratégicos
 
\item Artefato(s) Gerado(s) e/ou Alterado(s): Backlog de épicos
\item Atividade(s) Predecessora(s): Definir temas estratégicos
\item Atividade(s) Sucessora(s): Definir enablers de épicos

\end{itemize}

\subsection{Definir enablers dos épicos}
\begin{itemize}
\item Participantes: Epic owners

\item Descrição: De acordo com o SAFe, os épicos facilitadores(Enabler epics) são utilizados para dar suporte aos próximos épicos de negócio e evoluir a visão arquitetural do projeto. Com isso surgem boa parte dos requisitos não-funcionais que vão estar detalhados como enablers.

\item Subatividades(s):
	\begin{itemize} 
    \item Levantar Enablers: definir os enablers que agregarão valor significativo aos épicos já coletados.
    \item Analisar Enablers: Verificar se o time está a par desses enablers.
    \item Documentar Enablers:  Registrar esses enablers no Backlog de épicos.
    \end{itemize}
\item Artefatos de entrada: Backlog de épicos

\item Artefato(s) Gerado(s) e/ou Alterado(s): Backlog de épicos atualizado

\item Atividade(s) Predecessora(s): Definir épicos de negócio
  
\item Atividade(s) Sucessora(s): Definir features
\end{itemize}
