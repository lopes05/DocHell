\chapter{Introdução}

	A processo de contrução de um software não está apenas atrelado nas ferramentas, uma abordagem bem sucinta e detalhada é necessária tomada de decisões de acordo com a natureza do projeto. Segundo Pressman(1995), para que um projeto de software seja bem sucedido é necessário que alguns parâmetros sejam corretamente analisados, como por exemplo escopo do software, os riscos envolvidos, os recursos necessários, as tarefas a serem realizadas, os indicadores a serem acompanhados e a sistemática a ser seguida.

	Dentre os maiores dificuldades acerca do desenvolvimento de software estão (Manhães Teles, 2005):

\begin{itemize}	
	\item Gastos que superam o orçamento;
	\item Consumo de tempo que supera o cronograma;
	\item Funcionalidades que não resolvem os problemas dos usuários;
	\item Baixa qualidade dos sistemas desenvolvidos.
\end{itemize}

	Um estudo promovido pelo Departamento de Defesa dos Estados Unidos concluir que 75 \% de todos os grandes sistemas de software adaptados falham e que o motivo é o pobre gerenciamento por parte do desenvolvedor e cliente e não o desempenho técnico.

	Diante dessa situação foram analisados e estudados o contexto do cliente, abordagem da metodologia, elicitação de requisitos, análise de ferramentes que auxiliam a gerência, artefatos, papéis e outros fatores que determinam um processo de engenharia de requisitos para o cliente Padaria da Vila com o objetivo de elaborar um processo o qual atenda a necessidade apresentada pelo cliente.

\section{Visão Geral do Relatorio}

Esse relatório consiste na apresentação do modelo de processo a ser utilizado na segunda fase do trabalho da disciplina de Requisitos de Software, é consistido de seções que envolve a escolha e justificativa da abordagem, o processo modelado, suas atividades, artefatos e papéis, as técnicas de elicitação de requisitos a serem utilizadas, um cronograma de como foi construído este documento e a definição da ferramenta a ser utilizada para gerenciamento dos requisitos.

