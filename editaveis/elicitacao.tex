\chapter{Elicitação de requisitos}
A fase de elicitação de requisitos é uma fase de suma importância em qualquer processo de  desenvolivento de software. 
Esta fase consiste em levantar, expor e entender, da melhor forma possível, as funcionalidades técnicas e não técnicas do sistema,
por meio de técnicas de elicitação. É importante relembrar a má utilização destas técnicas, pois um processo mal conduzido 
resultará em perda de tempo implementando funcionalidades desnecessárias e por 
consequência, desperdício de capital investido no desenvolivento do software.
Não existe nenhum algoritmo que selecione de forma estática, a(s) técnica(s) de elicitação a serem utilizadas, 
estas devem ser definidas pela expertize e conhecimento técnico da equipe de ER, e revisadas pelo responsável para garantir
que serão eficientes.

  \subsection{Técnicas de elicitação}
  
  \subsection{Estrevistas orientadas}
  Esta técnica, bastante utilizadas no processo inicial de levantamento de informações sobre o sistema
  ,consiste na realização de entrevistas com perguntas pré-definidas pelos entrevistadores, afim de direcionar o 
  entrevistado a um assunto específico, sendo ideal para explorar assuntos específicos, em um curto período de tempo,
  pois o seguimento do roteiro impede que tanto o entrevistador quanto o(s) entrevistado(s) se percam nas respostas. Esta 
  técnica se apresenta de forma benéfica visto que, nossa cliente é uma universítária, se comunicando em uma linguagem 
  similar a nossa, facilitando a sim a comunicação e diminuindo a probabilidade de erros causados pela falta desta, além de
  possuir boa disponibilidade de tempo, visto que estuda em nosso Campus.
  Serão utilizadas algumas boas práticas de entrevistas, como: 
  - a definição do tempo de reunião, para que se consiga manter o foco
  e alta produtividade durante este período.
  - o envio prévio dos tópicos que serão abordados na entrevista, para que o entrevistado possa se preparar, e dar respostas mais
  completas sobre os temas discutidos.
  - a gravação das reuniões, para que, se houver necessidade, estas possam ser revisadas posteriormente.
  - a revisão das anotações e decisões tomadas durante a reunião.
  
  \subsection{Entrevistas abertas}
  Esta técnica consiste na realização de entrevistas abertas, onde somente o tópico é direcionado, mas a equipe de elicitação
  discute de forma aberta oque os stakeholders esperam dos sistemas. Segue alguma das boas práticas de entrevistas, sendo estas:
  - a definição do tempo de reunião, para que se consiga manter o foco
  e alta produtividade durante este período.
   - a gravação das reuniões, para que, se houver necessidade, estas possam ser revisadas posteriormente.
  - a revisão das anotações e decisões tomadas durante a reunião.
  
  \subsection{Observação do ambiente de trabalho}
  Esta técnica consiste em visitar o ambiente de trabalho, e, através da observação deste, obter informações de
  mais baixo nível sobre o ambiente, e a forma com que os futuros afetados pelo sistema que será desenvolvido interagem
  com o contexto de análise correspondente. É importante ainda, realizar esta observação do ambiente sem muito alarde,
  visto que a presença do observador pode e deve gerar mudança no comportamento dos funcionários.
  
  \subsection{Criação de cenário do programa}
  Esta técnica , de certa forma parecida com  a criação de um protótipo, consiste em criando um certo cenário,
  desenvolver o fluxo de funcionalidades do sistema, gastando assim, muito menos tempo que a criação de um protótipo.
  Será util para nós pois nosso sistema possui um escopo pequeno, não necessitando de um protótipo para expressar 
  suas funcionalidades.
  
  \subsection{Uso do programa feito pelo usuário}
  Esta técnica de elicitação de requisitos se parece com o processo de revisão de sprint
  onde, no desenvolvimento ágil, apresentamos para o PO o incremento e a parte funcional do sistema construído.
  O diferencial é que, no uso desta técnica, é solicitado ao PO que use o a parte funcional do sistema expressando, 
  em voz alta, oque ele está pensando sobre a funcionalidade em todos os aspectos, tanto sobre oque ele observa, quanto
  sobre as intenções dele em utilizar determinada função do software, desta forma, é obtido um feedback
  instantâneo sobre o software em geral, e pode-se entender também, 
  como funcionam os interesses de uso de determinada funcionalidade
  do sistema pelo usuário.
